\chapter{Background}
\section{OPC UA}
\subsection{Introduction}
In a proprietary systems, connected devices may use different communication protocols. Software used to inspect and operate this system must be aware of all these protocols. 
This requires additional costs and time spent on creating and maintaining such system. The complexity can be reduced by using the OPC UA standard for communication \cite{CPOpcTech}.

\subsection{Legacy OPC}
OPC is a data exchange standard for communication between multiple data sources. These sources may include factory devices, laboratory equipment, databases, test system fixtures. Set of standard interfaces were defined by OPC Foundation to allow access to any device compatible with OPC \cite{whatopc}.

The first protocol is reffered as OPC Classic and it used Microsoft's COM(Component Object Model)/DCOM(Distributed Component Object Model) technology. 
COM/DCOM enabled communication between processes in multiple languages for Windows operating system. 
COM is an interface standard designed for software components and it allows systems to be built from different software vendors, that are designed to use this interface
COM is providing a communication interface layer, allowing both local and remote procedure calls between processes. 
Distributed COM (DCOM) is an extension of COM designed for distributed applications. DCOM allows components to communicate across the network.\cite{COMDCOM, intrOPC}.

The OPC Classic specifications provided definitions for accessing data, alarms, events and historical data\cite{OPCCLas}:
\begin{itemize}
  \item OPC Data Access (DA) - exchange of data which includes values, time and quality information in real time,
  \item OPC Historical Data Access (HDA) - reading, updating, and subscribing to historical data and events,
  \item and OPC Alarms and Events (A\&E) - receiving alarms and responding to them, reveiving events.
\end{itemize}

Because of its reliance on Microsoft Windows platform, OPC Classic has some limitations like security issues or dependency on Windows. \cite{whyOPC}

{  
  \color{red}
  OPC Implementation? - jest to calkiem fajnie opisane w \cite{intrOPC}, ale nie wiem czy doda� bo mozliwe ze by�by to zbyt duzy offtop, bo uzywam opc ua.
}
\subsection{OPC UA}
In 2006 OPC Foundation released OPC Unified Architecture (UA) which is an improvement on OPC Classic. It brings together all the different specifications of OPC Classic into single entry point with DA, A\&E with HDA of both.

It is also based on a cross-platform Service-Oriented Architecture (SOA). SOA is an improvement on the security and functionality that was in OPC Classic with Microsofts COM/DCOM technology. 

OPC UA supports two protocols:
\begin{itemize}
  \item binary protocol - requires minimal resources, easy passing through firewall
  \item Simple Object Access Protocol (SOAP) - uses HTTP/HTTPS ports
\end{itemize}

Because of benefits of this protocol many industrial applications adopted OPC UA.

  % TODO Cross-Platform Capabilities
  % TODO Expanded security?
  % TODO IT Integration
  % TODO Compatibility
  % TODO applications
  % TODO client
  % TODO server
  
  {
    \color{red}
    How can it be used, where it's used?
    
    https://www.ni.com/pl-pl/innovations/white-papers/12/why-opc-ua-matters.html\\
  }

\section{Abstraction layers}
\subsection{Introduction}
Systems created currently are constructed using numerous abstraction layers. Each layer is defining an inteface that hides details of implementation for its functionality. Programs that are built on top of each layer can be used and understood using only its inteface, so it's not necessary to know the layer implementation.
\subsection{Specification}
Important aspect of an abstraction layer is its specification. A specification should capture the functionality of the implementation as well as the assuptions about other layer context. However, abstraction layers are not commonly specified and verified.
\subsection{Issues}
{
  \color{red}
  Write something about issues when creating abstraction layers\\
  
  sources: \\
  Deep Specifications and Certified Abstraction Layers \\
  http://flint.cs.yale.edu/flint/publications/dscal.pdf\\
  
  Separation Logic: A Logic for Shared Mutable Data Structures \\
  https://sci-hub.se/10.1109/LICS.2002.1029817\\
  https://ieeexplore.ieee.org/document/1029817\\
  
  Theories of Programming Languages\\
  John C. Reynolds\\
  
  Clean code\\
  str 36, 93, 290\\
}